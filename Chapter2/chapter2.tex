\chapter{Tổng quan về tóm tắt văn bản}
\label{Chapter2}

\section{Giới thiệu về tóm tắt văn bản}
\subsection{Giới thiệu chung}
Từ công trình đầu tiên của Luhn năm 1958, tóm tắt văn bản đã và đang trở thành một trong những tác vụ phổ biến và cần thiết nhất. Đặc biệt, trong sự bùng nổ không ngừng về công nghệ thông tin, việc tiếp nhận và xử lý khối lượng thông tin ngày càng lớn đang trở thành bài toán vô cùng thiết thực và quan trọng. Theo Luhn, mục đích của tóm tắt là nhằm tạo điều kiện giúp xác định nhanh chóng và chính xác chủ đề của văn bản gốc. Mục tiêu là tiết kiệm thời gian và công sức của người đọc trong việc tìm kiếm thông tin hữu ích của văn bản hoặc báo cáo ~\cite{Luhn58}
\cite{199-TED}

\subsection{Phân loại tóm tắt văn bản}

Bài toán tóm tắt văn bản được phân loại dựa trên nhiều khía cạnh khác nhau. Mỗi khía cạnh được áp dụng cho một mục đích khác nhau cũng như đòi hỏi các cách giải quyết khác nhau. Do đó, khó có một phương pháp chung nào có thể áp dụng tổng quát cho tất cả các loại tóm tắt văn bản. Vì vậy, cần xác định rõ đối tượng cũng như mục tiêu bài toán để chọn lựa phương pháp giải quyết cho phù hợp. Nhìn chung, tóm tắt văn bản có thể được chia theo một số dạng sau:
\begin{itemize}
	\item Về hình thức, tóm tắt văn bản được chia làm 2 loại: tóm tắt tóm lượt (abstractive) và tóm tắt rút trích (extractive). Theo đó, tóm tắt rút trích được tạo ra bằng các nối kết các câu được trích nguyên gốc từ văn bản ban đầu. Trong khi đó, tóm tắt tóm lượt được tạo ra bằng ngôn ngữ của người tóm tắt dựa trên nội dung của văn bản ban đầu. 
	\item Về đối tượng, tóm tắt văn bản được chia làm 2 loại: đơn văn bản (single document) và đa văn bản (multi documents). Với tóm tắt đơn văn bản, đầu vào của bài toán chỉ là một văn bản xác định. Khác với nó, tóm tắt đa văn bản nhận đầu vào là một tập các văn bản khác nhau mà có cùng chủ đề hoặc sự kiện.
	\item Về nội dung, tóm tắt văn bản được chia làm tóm tắt chỉ định (indicative) và tóm tắt thông tin (informative). Mục đích của tóm tắt chỉ định là giúp người đọc quyết định xem có nên tiếp tục đọc hay không bằng việc cung cấp các đặc trưng của văn bản như: chiều dài, văn phong, … Trong khi đó, tóm tắt thông tin cung cấp các sự kiện, vấn đề được tường thuật trong văn bản đầu vào.
	\item Về mục đích, tóm tắt văn bản được chia làm: tóm tắt tổng quát (generic) và tóm tắt hướng truy vấn (query focused). Tóm tắt tổng quát đặt ra giả thuyết độc giả là chung chung. Trong khi đó, mục đích của tóm tắt hướng truy vấn là tóm tắt các thông tin liên quan dựa trên một số yêu cầu truy vấn của người dùng. 
\end{itemize}


\section{Các phương pháp tóm tắt văn bản}



\section{Các vấn đề trong tóm tắt văn bản}

\subsection{Sắp xếp câu (Sentence ordering)}
Trong tóm tắt trích xuất, các câu quan trọng từ văn bản đầu vào sẽ được chọn lọc và đưa vào văn bản tóm tắt đầu ra. Tuy nhiên, văn bản là một chỉnh thể thống nhất có thứ tự của các câu văn. Vì thế, những câu quan trọng sau khi được chọn lọc cần phải trải qua việc tái sắp xếp để đảm bảo tính đúng đắn về mặt ngữ nghĩa. Một cách trực quan, sắp xếp câu nghĩa là tìm ra một trật tự có nghĩa của một tập câu cho trước sao cho văn bản tóm tắt phản ánh gần nhất văn bản đầu vào. 

\subsection{Giản lược câu (Sentence revision)}
Giản lược câu là việc biến đổi văn bản tóm tắt tạm thời bằng cách thay thế hoặc điều chỉnh các từ hoặc ngữ bằng từ hoặc ngữ khác thích hợp hơn dựa trên ngữ cảnh của văn bản tóm tắt. Về cơ bản, một số loại giản lược được đề xuất gồm : loại bỏ những thành phần (câu, ngữ, từ) không cần thiết, kết hợp các thông tin từ những câu khác nhau và thay đổi thành phần này bằng thành phần khác. Tuy nhiên, tác vụ này tương đối phức tạp và thường được nghiên cứu tập trung vào từng loại giản lược khác nhau.

\subsection{Kết hợp câu}

\subsubsection{Hợp nhất câu (Sentence fusion)}
Hợp nhất câu là tác vụ được thực hiện trên hai câu mà trong đó có sự trùng lắp một số thông tin. Mục đích của nó là tạo ra một câu mới chứa thông tin trùng lắp của các câu đầu vào.


\subsubsection{Nén câu (Sentence compression)}
Nhiều nhà nghiên cứu nhận thấy các văn bản tóm tắt thường chứa chỉ một phần của văn bản gốc. Những thành phần, yếu tố không cần thiết trong câu như bổ ngữ, mệnh đề phụ, chú giải, … thường được lượt bỏ nhằm giúp tăng tính xúc tích của văn bản tóm tắt. 


