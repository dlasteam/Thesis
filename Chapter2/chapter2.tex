\chapter{Trình bày luận văn}
\label{Chapter2}

Luận văn phải được trình bày ngắn gọn, rõ ràng, mạch lạc, sạch sẽ, không được tẩy xóa, có đánh số trang, đánh số bảng biểu, hình vẽ, đồ thị.
Luận văn được in trên một mặt giấy A4 (210 x 297 mm) và không vượt quá 80 trang A4, không tính phần phụ lục (nếu có).
Cỡ chữ Times New Roman 13 của hệ soạn thảo Winword hoặc tương đương.
Mật độ chữ bình thường, không được nén hoặc kéo dãn khoảng cách giữa các chữ; dãn dòng đặt ở chế độ 1,5 lines; lề trên 3,5cm; lề dưới 3 cm; lề trái 3,5cm; lề phải 2 cm.
Số trang được đánh ở giữa, phía cuối mỗi trang giấy.
Tất cả các chương, phần, mục, tiểu mục phải dùng chữ số tự nhiên, không được dùng số la mã.
Mẫu trình bày trang bìa (phụ lục 2), trang phụ bìa (phụ lục 3) và mục lục (phụ lục 4).

\section{Tiểu mục}

Các tiểu mục của luận văn được trình bày và đánh số thành nhóm chữ số, nhiều nhất gồm bốn chữ số với số thứ nhất chỉ số chương (ví dụ: 4.1.2.1. chỉ tiểu mục 1 nhóm tiểu mục 2 mục 1 chương 4).
Tại mỗi nhóm tiểu mục phải có ít nhất hai tiểu mục, nghĩa là không thể có tiểu mục 2.1.1 mà không có tiểu mục 2.1.2 tiếp theo.

\section{Bảng biểu, hình vẽ, phương trình}

Việc đánh số bảng biểu, hình vẽ, phương trình phải gắn với số chương; ví dụ hình 3.4 có nghĩa l hình thứ 4 trong Chương 3.
Mọi đồ thị, bảng biểu lấy từ các nguồn khác phải được trích dẫn đầy đủ. Nguồn được trích dẫn phải được liệt kê chính xác trong danh mục Tài liệu tham khảo.
Đầu đề của bảng biểu ghi phía trên bảng, đầu đề của hình vẽ ghi phía dưới hình.
Thông thường, những bảng ngắn và đồ thị phải đi liền với phần nội dung đề cập tới các bảng và đồ thị này ở lần thứ nhất.
Các bảng dài có thể để ở những trang riêng nhưng cũng phải tiếp theo ngay phần nội dung đề cập tới bảng này ở lần đầu tiên.
Các bảng rộng vẫn nên trình bày theo chiều đứng dài 297mm của trang giấy, chiều rộng của trang giấy có thể hơn 210mm.
Chú ý gấp trang giấy sao cho số và đầu đề của hình vẽ hoặc bảng vẫn có thể nhìn thấy ngay mà không cần mở rộng tờ giấy.
Tuy nhiên hạn chế sử dụng các bảng quá rộng này.

Đối với những trang giấy có chiều đứng hơn 297mm (bản đồ, bản vẽ,...) thì có thể để trong một phong bì cứng đính bên trong bìa sau của luận văn.
Các hình vẽ phải sạch sẽ bằng mực đen để có thể sao chụp lại; có đánh số và ghi đầy đủ đầu đề, cỡ chữ phải bằng cỡ chữ sử dụng trong văn bản luận văn.
Khi đề cập đến các bảng biểu và hình vẽ phải nêu rõ số của hình và bảng biểu đó, ví dụ ``... được nêu trong Bảng 4.1'' hoặc ``xem Hình 3.2'' mà không được viết ``… được nêu trong bảng dưới đây'' hoặc ``trong đồ thị của X và Y sau''.

Việc trình bày phương trình toán học trên một dòng đơn hoặc dòng kép tùy ý, tuy nhiên phải thống nhất trong toàn luận văn.
Khi ký hiệu xuất hiện lần đầu tiên thì phải giải thích và đơn vị tính phải đi kèm ngay trong phương trình có ký hiệu đó.
Nếu cần thiết, danh mục của tất cả các ký hiệu, chữ viết tắt và nghĩa của chúng cần được liệt kê và để ở phần đầu của luận văn.
Tất cả các phương trình cần được đánh số và để trong ngoặc đơn đặt bên phía lề phải.
Nếu một nhóm phương trình mang cùng một số thì những số này cũng được để trong ngoặc, hoặc mỗi phương trình trong nhóm phương trình (5.1) có thể được đánh số là (5.1.1), (5.1.2), (5.1.3).

\section{Viết tắt}

Không lạm dụng việc viết tắt trong luận văn.
Chỉ viết tắt những từ, cụm từ hoặc thuật ngữ được sử dụng nhiều lần trong luận văn.
Không viết tắt những cụm từ  dài, những mệnh đề; không viết tắt những cụm từ ít xuất hiện trong luận văn.
Nếu cần viết tắt những từ thuật ngữ, tên các cơ quan, tổ chức,... thì được viết tắt sau lần viết thứ nhất có kèm theo chữ viết tắt trong ngoặc đơn.
Nếu luận văn có nhiều chữ viết tắt thì phải có bảng danh mục các chữ viết tắt (xếp theo thứ tự ABC) ở phần đầu luận văn.

\section{Tài liệu tham khảo và cách trích dẫn}

Mọi ý kiến, khái niệm có ý nghĩa, mang tính chất gợi ý không phải của riêng tác giả và mọi tham khảo khác phải được trích dẫn và chỉ ra nguồn trong danh mục tài liệu tham khảo của luận văn.
Không trích dẫn những kiến thức phổ biến, mọi người đều biết cũng như không làm luận văn nặng nề với những tham khảo trích dẫn.
Việc trích dẫn, tham khảo chủ yếu nhằm thừa nhận nguồn của những ý tưởng có giá trị giúp người đọc theo được mạch suy nghĩ của tác giả, không làm trở ngại việc đọc.
Nếu không có điều kiện tiếp cận được một tài liệu gốc mà phải trích dẫn thông qua một tài liệu khác thì phải nêu ra trích dẫn này, đồng thời tài liệu gốc đó không được liệt kê trong danh mục tài liệu tham khảo của luận văn.
Cách sắp xếp danh mục tài liệu tham khảo xem phụ lục 5.
Việc trích dẫn là theo số thứ tự của tài liệu ở danh mục tài liệu tham khảo và được đặt trong ngoặc vuông, khi cần có cả số trang, ví dụ [15, tr.314-315].
Đối với phần được trích dẫn từ nhiều nguồn tài liệu khác nhau, số của từng tài liệu được đặt độc lập trong từng ngoặc vuông, theo thứ tự tăng dần, ví dụ [19], [25], [41], [42].

\section{Phụ lục của luận văn}

Phần này bao gồm nội dung cần thiết nhằm minh họa hoặc hỗ trợ cho nội dung luận văn như số liệu, mẫu biểu, tranh ảnh,...
Phụ lục không được dày hơn phần chính của luận văn.
